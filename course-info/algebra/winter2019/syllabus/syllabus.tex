\documentclass[11pt]{article}
\usepackage{amsmath}
\usepackage{amssymb}
\usepackage [usenames,dvipsnames]{color}
\def\ds{\displaystyle}
\def\ul{\underline}
\def\mb{\mathbf}
\usepackage{graphicx}
\usepackage{tikz}
\usepackage{enumitem}
\oddsidemargin=-0.0cm
\evensidemargin=-0.0cm
\topmargin=-0.0cm
\textwidth=7in
\textheight=9in
\newtheorem{theorem}{Theorem}
\def\ds{\displaystyle}
\def\ul{\underline}
\def\mb{\mathbf}
 

%%%%%%%%%%%%%%%%%%%%%%%%%%%%%%%%%%%%%%%%%%%%%%%%%%%%%%%%%%%%%%%%%%

\title{Math 398, Group Theory with a hint of Categories\vspace{-2ex}}
\pagestyle{myheadings}
\author{}
\date{}

\begin{document}
\maketitle
\markboth{Math 398}{Math 398}


% \date{\today}

%\vskip .1in
\hrule height 1pt
\vskip .1in

\begin{tabular}{ll}
Lecture:& Monday 12:30-1:20, \textit{Place TBD}\\
Faculty Sponsor: & TBD (hopefully Jim?)\\ 
Web address: &
\texttt{http://untyped.me/algebra.html}\\ 
Problem Section: & Friday 12:30-1:20, \textit{Place TBD}\\
Advanced Student Mentor: & Thomas Browning\\
E-mail: &\texttt{thomas.l.browning@gmail.com} \\
Text: & {\em Algebra: Chapter 0} \\
Author: & Paolo Aluffi\\
\end{tabular}

\vskip .2in


The intent of this student organized math study group is to cover the basics of group theory, from a categorical perspective. The focus will be similar to Math 402, but with a slightly more abstract bent. We will use techniques common in higher algebra courses much earlier than normal, with the aim of preparing students for the graduate algebra sequence, Math 504-5-6. We do not, however, assume any prior familiarity with algebra, beyond an extremely minimal amount of linear algebra. If this quarter is successful, we may try to cover Rings and Modules in the Spring. 

We will meet twice a week, every week. Students will be expected to read through the assigned section of Aluffi before the first meeting every week. During the first meeting each week, one to two students will present the material from that week's section to the class. Choosing to present during a certain week is voluntary, but all students are expected to present at least once. During the second meeting, students will present solutions to the assigned problems, like in Math 33X or Math 380. Problem sets will be generated by Thomas Browning, an advanced undergraduate student acting as a mentor for this study group. Problem sets will be posted on the course website.

Over Winter quarter, we plan to cover all of Chapters 2 and 4 from Aluffi, following this schedule:
\begin{enumerate}[start = 0, label = Week \arabic*:]
\item The language and basic concepts of Category Theory (covered before winter quarter)
\item The definition of a group and examples
\item Group homomorphisms and the category \textsf{Grp} of groups
\item Subgroups, Normality, and Quotient Groups
\item Canonical Decomposition and Lagrange's Theorem, Presentations and Free Groups
\item Group actions, the conjugation action, and the Class Formula
\item The Sylow Theorems
\item Composition Series and Solvability
\item The Symmetric Group
\item More products of groups, exact sequences of groups, the Extension Problem
\item The Classification of Finite Abelian Groups
\end{enumerate}

\pagenumbering{gobble}

\end{document}


